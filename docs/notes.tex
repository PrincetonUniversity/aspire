\documentclass{article}

\usepackage{amsfonts}
\usepackage{mathtools}

\usepackage{algorithm}
\usepackage{algpseudocode}
\usepackage{algorithmicx}

\newcommand{\euler}{\mathrm{e}}
\newcommand{\imunit}{\mathrm{i}}

\newcommand{\Natural}{\mathbb N}
\newcommand{\Integer}{\mathbb Z}
\newcommand{\Real}{\mathbb R}
\newcommand{\Complex}{\mathbb C}
\newcommand{\SO}{\mathrm{SO}}

\newcommand{\meshu}{M}
\newcommand{\meshc}{\overline{M}}

\newcommand{\sz}{L}

\newcommand{\vol}{x}
\newcommand{\im}{y}

\newcommand{\proj}{P}

\newcommand{\volfun}{\mathcal{X}}
\newcommand{\imfun}{\mathcal{Y}}

\newcommand{\projfun}{\mathcal{P}}

\newcommand{\vu}{\boldsymbol{u}}
\newcommand{\vomega}{\boldsymbol{\omega}}
\newcommand{\vn}{\boldsymbol{n}}
\newcommand{\vk}{\boldsymbol{k}}

\newcommand{\vc}{\boldsymbol{c}}

\newcommand{\vones}{\boldsymbol{1}}

\newcommand{\rot}{R}

\newcommand{\transp}{\mathrm{T}}

\newcommand{\fourier}[1]{\mathcal{F}#1}
\newcommand{\ifourier}[1]{\mathcal{F}^{-1}#1}

\begin{document}

\section{Projection}

Given a function $\volfun \in L^1(\Real^3)$, we define its projection $\projfun \volfun \in L^1(\Real^2)$ along $\rot \in \SO(3)$ as
\begin{equation}
	\projfun \volfun(\vu) \coloneqq \int_\Real \volfun(\rot^{-1} [\vu; z]) dz\mbox{,}
\end{equation}
where $[\vu; z] \in \Real^3$ is the concatenation of $\vu$ with $z$.
Let us define the $d$-dimensional Fourier transform $\fourier{\mathcal{H}} \in L^\infty(\Real^d)$ of $\mathcal{H} \in L^1(\Real^d)$ by
\begin{equation}
    \fourier{\mathcal{H}}(\vomega) \coloneqq \int_{\Real^d} \mathcal{H}(\vu) \euler^{-2\pi\imunit \langle \vomega, \vu \rangle} d\vu\mbox{.}
\end{equation}
The projection mapping $\projfun$ then satisfies the following identity,
\begin{equation}
	\label{eq:fourier-slice}
    \fourier{\projfun \volfun}(\vomega) = \fourier{\volfun}\left(\rot^{-1}[\vomega; 0]\right) \quad \forall \vomega \in \Real^2 \mbox{,}
\end{equation}
known as the Fourier slice theorem.
That is, projection in the spatial domain corresponds to restriction in the frequency domain.
This identity allows us to exploit fast Fourier transforms and non-uniform fast Fourier transforms in order to define fast discrete approximations of $\projfun$.

In the following, we shall use the following notation to describe the non-symmetric one-dimensional mesh of length $L$
\begin{equation}
    \meshu_L \coloneqq \{-\lfloor L/2 \rfloor, \ldots, \lceil L/2-1 \rceil\}\mbox{.}
\end{equation}
Note that we have
\begin{equation}
	\meshu_L =
	\left\{
	\begin{array}{ll}
		\{-(L-1)/2, \ldots, (L-1)/2\} & $L$ \mbox{~odd} \\
		\{-L/2, \ldots, L/2-1\} & $L$ \mbox{~even}
	\end{array}
	\right.
	\mbox{.}
\end{equation}
To obtain higher-dimensional meshes, we form the tensor product of $\meshu_L$ with itself, obtaining $\meshu_L^d \subset \Integer^d$.

The above mesh is quite natural for defining the discrete Fourier transform, as we shall see below.
However, in the case of even $L$, it is not symmetric about zero, which leads to difficulties when discretizing the projection mapping $\projfun$.
To remedy this, we define the symmetric one-dimensional mesh
\begin{equation}
	\meshc_L =
	\left\{
	\begin{array}{ll}
		\meshu_L & $L$ \mbox{~odd} \\
		\meshu_L + \frac{1}{2} & $L$ \mbox{~even}
	\end{array}
	\right.
	\mbox{.}
\end{equation}
Again, we define higher-dimensional meshes by forming the tensor product $\meshc_L^d$.
These meshes now have the property that $-\meshc_L^d = \meshc_L^d$ for all $L \ge 1$.

Discrete volumes and images will be defined as functions on these grids.
In general, such a function is given by $h: \meshu_L^d \rightarrow \Real$ or $h: \meshc_L^d \rightarrow \Real$ depending on whether it is defined on the non-symmetric or symmetric mesh, respectively.
We denote the values of these functions by $h[\vn]$ to distinguish them from continuous functions.
In some situations, it will be useful to consider the functions as column vectors of dimension $L^d$, that is, elements of $\Real^{L^d}$.

The discrete Fourier transform (DFT) on the $d$-dimensional mesh $\meshu_L^d$ of size $L$ is given by
\begin{equation}
	\label{eq:dft-def}
	\fourier{h}[\vk] = \sum_{\vn \in \meshu_L^d} h[\vn] \euler^{-2\pi\imunit \langle \vn, \vk \rangle / L} \quad \forall \vk \in \meshu_L^d \mbox{,}
\end{equation}
for some discrete function $h: \meshu_L^d \rightarrow \Real$.
The inverse DFT is then given by
\begin{equation}
	\label{eq:idft-def}
	\ifourier{g}[\vn] = \frac{1}{L^d} \sum_{\vk \in \meshu_L^d} g[\vk] \euler^{2\pi\imunit \langle \vn, \vk \rangle / L} \quad \forall \vn \in \meshu_L^d \mbox{.}
\end{equation}
As its name implies, the inverse DFT inverts the DFT and we have $\ifourier{\fourier{h}} = h$ for all $h: \meshu_L^d \rightarrow \Real$.
The DFT and the inverse DFT are calculated efficiently using the fast Fourier transform (FFT) algorithm.
Specifically, \eqref{eq:dft-def} and \eqref{eq:idft-def} are both computed in $O(L^d \log L)$ time.

The final tool we shall make use of is the non-uniform discrete Fourier transform (NUDFT).
Given a function $h: \Omega \rightarrow \Real$ defined on some discrete set $\Omega \subset \Real^d$ and a set of $N$ frequency nodes $\{\vomega_j\}_{j=1}^N \subset [-1/2, 1/2]^d$, it is defined as
\begin{equation}
	\fourier{h}(\vomega_j) = \sum_{\vu \in \Omega} h[\vu] \euler^{-2\pi\imunit \langle \vu, \vomega_j \rangle} \quad 1 \le j \le N\mbox{.}
\end{equation}
Note that we have overloaded the notation $\fourier$ to signify both the standard (grid-based) DFT as well as the NUDFT.
While the DFT is indexed by square brackets $[]$, we index the NUDFT with parentheses $()$.
We also define the adjoint NUDFT.
Applied to a set of coefficients $\vc = \{c_j\}_{j=1}^N$ for the same set of frequency nodes and domain $\Omega$, it is given by
\begin{equation}
	\fourier^*{\vc}[\vu] = \sum_{j=1}^N c_j \euler^{2\pi\imunit \langle \vu, \vomega_j \rangle} \quad \forall \vu \in \Omega\mbox{.}
\end{equation}
Note that, in general, the adjoint NUDFT does not invert the NUDFT.

Calculating the NUDFT or its adjoint directly has complexity $O(|\Omega| J)$.
However, when $\Omega = \meshu_L^d$, it can be accurately approximated using the non-uniform fast Fourier transform (NUFFT) algorithm.
The NUFFT has complexity $O(L^d \log L + J \log (1/\epsilon))$ for a desired accuracy $\epsilon$.

\subsection{Odd size $L$}

As it is simpler than the even case, we start with the case of odd-sized volumes and images.
For $L$ odd, we thus have a volume of size $L$-by-$L$-by-$L$ which we would like to project along some rotation $\rot \in \SO(3)$ onto an image of size $L$-by-$L$.
Although there is no difference between the non-symmetric and symmetric meshes in the case of odd $L$, we shall define the volume over $\meshc_L^3$ for consistency with the even case.
From this, we would like images defined on $\meshc_L^2$.

In order to exploit the efficiency of the FFT and the NUFFT, we will define our discrete projection mapping to satisfy a discrete version of the Fourier slice theorem \eqref{eq:fourier-slice}.
Specifically, we require that the projection mapping $\proj$ along $\rot$ satisfy
\begin{equation}
	\fourier{\proj \vol}(\vk/L) = \fourier{\vol}(\rot^{-1} [\vk; 0]/L) \quad \forall \vk \in \meshc_L^d\mbox{.}
\end{equation}
That is the NUDFT at grid point in $\vk \in \meshc_L^d$ of the projection $\proj \vol$ should equal the NUDFT of the volume $\vol$ on the grid given by $\rot^{-1} [\vk; 0]/L$.

Since $\meshc_L = \meshu_L$ for the odd case, we have that the left-hand side is actually a regular DFT $\fourier{\proj \vol}[\vk]$ which can be inverted by the IDFT to yield
\begin{equation}
    \proj \vol[\vn] = \ifourier{(\fourier{\vol}(\rot^{-1} [\cdot; 0]/L))}[\vn]\mbox{.}
\end{equation}
A fast algorithm for calculating $\proj \vol$ is then to first compute the NUFFT of $\vol$ on the grid $\rot^{-1} [\vk; 0]/L$, then perform the IFFT.
This is algorithm is detailed in Algorithm \ref{algo:odd-proj}.

\begin{algorithm}[t]
\begin{algorithmic}
\Function{ProjectOdd}{$\rot$, $\vol: \meshc_L^3 \rightarrow \Real$}
\State $s[\vk] \gets \fourier{\vol}(\rot^{-1} [\vk; 0]/L)$ for all $\vk \in \meshu_L^2$ using the NUFFT.
\State $\proj\vol[\vn] = \ifourier{s}[\vn]$ for all $\vn \in \meshu_L^2$ using the IFFT.
\EndFunction
\end{algorithmic}
\caption{
\label{algo:odd-proj}
Projection for odd $L$.
}
\end{algorithm}

The complexity of the NUFFT is $O(L^3 \log L + L^2) = O^(L^3 \log L)$ while the IFFT takes $O(L^2 \log L)$.
As a result, the overall complexity is $O(L^3 \log L)$.
If we wish to compute the projection of $\vol$ along $N$ different rotations, the complexity is $O(L^3 \log L + N L^2 \log L)$.

\subsection{Even size $L$}

The even-sized images require a more careful approach to avoid problems with the non-symmetric grid $\meshu_L$.
As in the odd-sized case, we wish to have our projection mapping $\proj$ along $\rot$ satisfy
\begin{equation}
	\fourier{\proj \vol}(\vk/L) = \fourier{\vol}(\rot^{-1} [\vk; 0]/L) \quad \forall \vk \in \meshc_L^d\mbox{.}
\end{equation}
However, since $\meshc_L \neq \meshu_L$, the left-hand side no longer corresponds to a standard DFT.
In addition, the right-hand side NUDFT is defined over $\meshc_L^3$, while the NUFFT is defined for the non-symmetric grid $\meshu_L^3$.
To obtain an efficient algorithm for computing $\proj\vol$, we need to manage these problems.

Let us first consider the right-hand side.
Since $\vol$ is defined on $\Omega = \meshc_L^3$, we have
\begin{align}
    &\fourier{\vol}(\rot^{-1} [\vk; 0]/L) \\
    &\quad = \sum_{\vn \in \meshc_L^3} \vol[\vn] \euler^{-2\pi\imunit \langle \vn, \rot^{-1} [\vk; 0]\rangle/L} \\
    &\quad = \sum_{\vn \in \meshu_L^3} \vol[\vn+1/2] \euler^{-2\pi\imunit \langle \vn+1/2, \rot^{-1} [\vk; 0]\rangle/L} \\
    &\quad = \euler^{-2\pi\imunit \langle \vones, \rot^{-1} [\vk; 0] \rangle/2L} \sum_{\vn \in \meshu_L^3} \vol[\vn+1/2] \euler^{-2\pi\imunit \langle \vn, \rot^{-1} [\vk; 0] \rangle/L} \mbox{,}
\end{align}
where $\vones = [1; 1; 1]$, since $\meshc_L^3 = \meshu_L^3 + 1/2$.
The right-hand side is therefore computed using an NUFFT followed by a phase shift.

\begin{algorithm}[t]
\begin{algorithmic}
\Function{ProjectEven}{$\rot$, $\vol: \meshc_L^3 \rightarrow \Real$}
\State $\widetilde{\vol}[\vn] \gets \vol[\vn+1/2]$ for all $\vn \in \meshu_L^3$.
\State $s[\vk] \gets \fourier{\widetilde{\vol}}(\rot^{-1} [\vk; 0]/L)$ for all $\vk \in \meshc_L^2$ using the NUFFT.
\State $s[\vk] \gets \euler^{-2\pi\imunit \langle \vones, \rot^{-1} [\vk; 0] \rangle/2L} s[\vk]$ for all $\vk \in \meshc_L^2$.
\State $s[\vk] \gets \euler^{2\pi\imunit \langle \vones, \vk \rangle/2L} s[\vk]$ for all $\vk \in \meshc_L^2$.
\State $\widetilde{s}[\vk] \gets s[\vk+1/2]$ for all $\vk \in \meshu_L^2$.
\State $\widetilde{\proj\vol}[\vn] \gets \ifourier \widetilde{s}[\vn]$ for all $\vn \in \meshu_L^2$ using the IFFT.
\State $\widetilde{\proj\vol}[\vn] \gets \euler^{2\pi\imunit \langle \vn, \vones \rangle/2L} \widetilde{\proj\vol}[\vn]$ for all $\vn \in \meshu_L^2$.
\State $\proj\vol[\vn] \gets \widetilde{\proj\vol}[\vn-1/2]$ for all $\vn \in \meshc_L^2$.
\EndFunction
\end{algorithmic}
\caption{
\label{algo:even-proj}
Projection for even $L$.
}
\end{algorithm}

The left hand side at $\vk \in \meshc_L^3$ is given by
\begin{align}
    &\fourier{\proj \vol}(\vk/L) \\
    &\quad = \sum_{\vn \in \meshc_L^2} \proj\vol[\vn] \euler^{-2\pi\imunit \langle \vn, \vk \rangle/L} \\
    &\quad = \sum_{\vn \in \meshu_L^2} \proj\vol[\vn+1/2] \euler^{-2\pi\imunit \langle \vn+1/2, \vk \rangle/L} \\
    &\quad = \euler^{-2\pi\imunit \langle \vones, \vk \rangle/2L} \sum_{\vn \in \meshu_L^2} \proj\vol[\vn+1/2] \euler^{-2\pi\imunit \langle \vn, \vk \rangle/L}\mbox{,}
\end{align}
since $\meshc_L^2 = \meshu_L^2 + 1/2$.
The sum is still not a DFT since $\vk \in \meshc_L^2$.
However, $\vk-1/2$ is in $\meshu_L^2$, so we can rewrite it as
\begin{align}
    &\sum_{\vn \in \meshu_L^2} \proj\vol[\vn+1/2] \euler^{-2\pi\imunit \langle \vn, \vk \rangle/L} \\
    &\quad = \sum_{\vn \in \meshu_L^2} \proj\vol[\vn+1/2] \euler^{-2\pi\imunit \langle \vn, (\vk-1/2)+1/2 \rangle/L} \\
    &\quad = \sum_{\vn \in \meshu_L^2} \left(\proj\vol[\vn+1/2] \euler^{-2\pi\imunit \langle \vn, \vones \rangle/2L}\right) \euler^{-2\pi\imunit \langle \vn, (\vk-1/2)\rangle/2} \mbox{.}
\end{align}
This sum is a DFT and can therefore be efficiently inverted using the IFFT.

Putting all of this together, we first compute an NUFFT, apply a phase shift, then apply the IFFT, followed by another phase shift.
The complete algorithm is outlined in Algorithm \ref{algo:even-proj}.

The only additional steps compared to Algorithm \ref{algo:odd-proj} are the phase shifts, each of which has computational complexity $O(L^2)$.
Consequently, the overall complexity remains $O(L^3 \log L)$ to project along a single rotation while the complexity for projecting along $N$ distinct rotations is $O(L^3 \log L + N L^2 \log L)$.

\end{document}
