\documentclass[12pt]{article}

\usepackage{amsmath}
\usepackage{amsfonts}
\usepackage{amssymb}

%included packages (some not so standard, let me know if you don't want to use them)
\usepackage[left=2cm,right=2cm,top=2cm,bottom=2cm]{geometry}

\usepackage[dvipdf,colorlinks=true,citecolor=blue,linkcolor=blue]{hyperref}
%\usepackage[pdftex,colorlinks=true,citecolor=blue,linkcolor=blue]{hyperref}

%\usepackage[authoryear,round]{natbib}

\usepackage{algorithm}
\usepackage{algorithmic}
\usepackage{url}
\usepackage{graphics}
\usepackage{verbatim}

\newtheorem{theorem}{Theorem}[section]
\newtheorem{proposition}[theorem]{Proposition}
\newtheorem{lemma}[theorem]{Lemma}
\newtheorem{example}[theorem]{Example}
\newtheorem{corollary}[theorem]{Corollary}
\newtheorem{assumption}[theorem]{Assumption}
\def\proof{\noindent{\bf Proof.} \ignorespaces}
\def\endproof{{\ \hfill\hbox{%
      \vrule width1.0ex height1.0ex
    }\parfillskip 0pt}\par}

\marginparwidth 0pt \oddsidemargin  0pt \evensidemargin  0pt
\marginparsep 0pt

\setlength{\textwidth}{6.5in} \setlength{\textheight}{9in}
\setlength{\topmargin}{-.45in}
\setlength{\evensidemargin}{-0.14in}
\setlength{\oddsidemargin}{-0.14in}
%\renewcommand{\baselinestretch}{1.2}

\newcommand{\vgap}{\vspace*{.1in}}
\newcommand{\bu}{\bullet}
\newcommand{\A}{{\cal A}}
\newcommand{\K}{{\cal K}}
\newcommand{\Sn}{{\cal S}^n}
\newcommand{\Snp}{{\cal S}^n_+}
\newcommand{\Snpp}{{\cal S}^n_{++}}
\newcommand{\beq}{\begin{equation}}
\newcommand{\eeq}{\end{equation}}
\newcommand{\beqa}{\begin{eqnarray}}
\newcommand{\eeqa}{\end{eqnarray}}
\newcommand{\beqas}{\begin{eqnarray*}}
\newcommand{\eeqas}{\end{eqnarray*}}
\newcommand{\rank}{{\rm rank}\,}
\newcommand{\Image}{{\rm Im}\,}
\def\eqnok#1{(\ref{#1})}
\newcommand{\mdot}{\bullet}
\newcommand{\bSn}{{\cal S}^{n_1 \cdots n_\ell}}
\newcommand{\bSnp}{{\cal S}^{n_1 \cdots n_\ell}_+}
\newcommand{\tr}{\operatorname{trace}}
\newcommand{\eqnref}[1]{(\ref{#1})}
\newcommand{\thmref}[1]{Theorem \ref{#1}}
\newcommand{\sectref}[1]{Section \ref{#1}}
\newcommand{\propref}[1]{Proposition \ref{#1}}
\newcommand{\blockRspace}{\Re^{(n_1 \times r_1)\cdots(n_\ell \times
r_\ell)}}
\renewcommand{\L}{{\cal L}}
\newcommand{\blas}{{\sc blas}}
\newcommand{\blass}{{\sc blas }}
\newcommand{\lapack}{{\sc lapack}}
\newcommand{\lapacks}{{\sc lapack }}
\newcommand{\arpack}{{\sc arpack}}
\newcommand{\arpacks}{{\sc arpack }}
\newcommand{\metis}{{\sc metis}}
\newcommand{\metiss}{{\sc metis }}

\begin{document}

\title{SDPLR 1.03-beta User's Guide (short version)}
%\author{
%Samuel Burer%
%\thanks{Department of Management Sciences, University of Iowa,
%Iowa City, IA 52242-1000, USA. (Email: samuel-burer@uiowa.edu).
%This research was supported in part by NSF Grant CCR-0203426.}
%\and Renato D.C.\ Monteiro%
%\thanks{Department of Industrial and Systems Engineering, Georgia Institute
%of Technology, Atlanta, GA 30332, USA. (Email:
%monteiro@isye.gatech.edu).} \and Changhui Choi%
%\thanks{Department of Management Sciences, University of Iowa,
%Iowa City, IA 52242-1000, USA. (Email: changhui-choi@uiowa.edu).
%This research was supported in part by NSF Grant CCR-0203426.}}

\date{\today}

\maketitle

\section{Summary}

SDPLR is a C software package for solving large-scale semidefinite
programming problems. Source code, binaries, and a Matlab
interface for SDPLR can be downloaded from
\begin{center}
\url{http://dollar.biz.uiowa.edu/~sburer/software/SDPLR/}.
\end{center}
SDPLR accepts the standard sparse SDPA format as well as a special
format that handles low-rank data matrices. People involved with
the development of SDPLR are
\begin{itemize}

\item Samuel Burer (samuel-burer@uiowa.edu)

\item Renato D.C.\ Monteiro (monteiro@isye.gatech.edu)

\item Changhui Choi (changhui.choi@ucdenver.edu)

\end{itemize}

% \section{Binaries and Installation from Source}
\section{Installation from Source}

% Binaries for different platforms are available from the SDPLR
% website. Users' contribution of binaries would be especially
% appreciated. In the event that no binary is available for a
% particular platform, the source code is also available for
% installation from scratch.

It is assumed that the user has downloaded and unpacked the SDPLR
source files.

\subsection{Matlab}

Within Matlab, run {\sl mexinstall.m} from the main SDPLR directory.
Then type {\sl help sdplr} for instructions. Add the SDPLR directory to
your Matlab path.

{\bf Note to Windows users.} The top two lines of {\sl mexinstall.m} may
need to be adjusted for the location of the Matlab's \lapack~ and \blas~
libraries on your system.

\subsection{Command line (UNIX, Linux, etc.)}

\subsubsection*{Step 1}

Make sure \blass and \lapacks are installed on your system and
determine how to link them with other programs. The linking
information will be required for the {\tt LIB} and {\tt LIB\_DIRS}
options in {\sl Makefile.inc\/} (see Step 2).

% Many systems have
% optimized \blass and \lapacks libraries.

% On Linux, Cygwin, and
% Mingw systems, optimized libraries for common architectures can be
% downloaded from
% \begin{center}
% %\url{http://www.scipy.org/download/atlasbinaries/}.
% \url{http://old.scipy.org/download/atlasbinaries/}.
% \end{center}

% \subsubsection{Step 2 (optional)}

% If you would like the ability to use some of SDPLR's advanced
% features, including sparse eigenvalue computations and
% preconditioning for the truncated Newton's method, download and
% compile
% \begin{center}
% \arpack \ \ \ \url{http://www.caam.rice.edu/software/ARPACK/}
% \end{center}
% and
% \begin{center}
% \metis 4.0.1 \ \ \
% \url{http://www-users.cs.umn.edu/~karypis/metis/metis/index.html}.
% \end{center}
% Both compilations produce static libraries (.a), which can be
% copied into the {\sl lib\/} directory of the SDPLR package.
% Whether \arpacks and/or \metiss are available will be required for
% the {\tt DEFINES} option in {\sl Makefile.inc\/} (see Step 3).

\subsubsection*{Step 2}

Tailor the file {\sl Makefile.inc\/} to your system; {\sl
Makefile.inc.linux\/} is provided as an example. There are three
sections:

\begin{itemize}

\item Specify the location and switch for the \lapack~library
using {\tt LAPACK\_LIB\_DIR} and {\tt LAPACK\_LIB}.

\item Specify the location and switch for the \blas~library
using {\tt BLAS\_LIB\_DIR} and {\tt BLAS\_LIB}.

\item Specify the compiling options:

\begin{itemize}

\item {\tt CC}: Specify the compiler, e.g., {\sl gcc} or {\sl
cc}.

\item {\tt CFLAGS}: Specify any flags to pass to the compiler and
linker, e.g., optimization flags such as {\tt -O3}.

\item {\tt LIB\_DIRS}: If necessary, using the {\tt -L} switch, specify
any directory locations for additional libraries that need to be
linked in. Leave items involving \lapack and \blas~ unchanged. It is
not necessary to include the SDPLR {\sl lib} directory since it is
automatically specified.

\item {\tt LIBS}: Specify the flags for the libraries that will be
linked in. This should definitely include {\tt -lgsl}, which is included
with and required by SDPLR, and the flags for the \blass and \lapacks
libraries from Step 1. Remember that linking order is important! For
example, on an Ubuntu 9.04 Linux system, the line might read
\begin{center}
{\tt LIBS = -lgsl -llapack -lblas -lgfortran -lm}
\end{center}
Here, {\tt -lgfortran} indicates a library that is required by \lapacks
and \blas, and {\tt -lm} is the standard math library.

\end{itemize}

% \item {\tt DEFINES}: If \arpacks and \metiss are to be linked in,
% then {\tt -D\_\_ARPACK} and {\tt -D\_\_METIS} should be specified.
% (The {\tt \_\_} symbol is two underscore characters.)

% \item {\tt  MEX}: (optional) If  you will  be using SDPLR  under Matlab,
% specify the Mex binary here.

% \item {\tt MEXEXT}: (optional) If you  will be using SDPLR under Matlab,
% specify the  default Matlab  binary extension  for your  platform (e.g.,
% {\tt mexglx} under Linux).

% \item {\tt LIBSMEX}: (optional) If you  will be using SDPLR under Matlab,
% specify those libraries which Mex requires. These may be the same or
% different from those specified in {\tt LIBS}.

\end{itemize}

\subsubsection*{Step 3}

Type {\sl make} to compile the program. Then type {\sl sdplr
vibra1.dat-s} to test the installation. Your output should be similar to
the file {\sl vibra1.dat-s.out} provided in the installation directory.
Typing {\sl make clean} will remove all object files, and typing {\sl
make cleanall} will remove the SDPLR executable and GSL library as well.

% \subsubsection{Step 5 (optional)}

% If you would like to use SDPLR from within Matlab, type {\sl make
% mex}; this assumes that the Matlab MEX compiler is called `mex'.
% The appropriate Matlab binary will be created. Here again, typing
% {\sl make clean} will remove all object files, and typing {\sl
% make cleanall} will remove the SDPLR executable, the GSL library,
% and the Matlab binary.

\subsection{Windows (via MinGW)}

MinGW is a GNU compiling environment for Windows. To compile SDPLR, you will need

\begin{itemize}

\item The MinGW base system (including the {\sl make\/} utility). Add
MinGW's {\sl bin\/} directory to your Windows path.

\item MinGW's {\sl gcc\/} tools. It is suggested to install the latest
{\sl gcc\/} release with additional packages such as {\sl gfortran\/}.

\item Working \lapack~ and \blas~ libraries within MinGW. The following
guide is suggested: {\tt http://gcc.gnu.org/wiki/LAPACK\%20on\%20Windows}.

\end{itemize}

Then tailor {\sl Makefile.inc\/} as described above; {\sl
Makefile.inc.mingw} is provided as an example. Finally, type
{\sl mingw32-make mingw\/}.

\section{Usage}


\subsection{Matlab}

Make sure that the Matlab binary for SDPLR and the file {\sl sdplr.m\/}
are in your Matlab path. Then type {\sl help sdplr} for more
information.

\subsection{Command line}

The syntax for SDPLR is
\begin{quote}
Usage \#1: {\tt sdplr <input\_file> [params\_file] [soln\_in]
[soln\_out]} \\ Usage \#2: {\tt sdplr gen\_params}
\end{quote}

\noindent Regarding usage \#1:

\begin{itemize}

\item If {\tt params\_file} is not specified, then a set of
default parameters is used.

\item Both files {\tt soln\_in} and {\tt soln\_out} refer to a
particular file format generated and used by SDPLR.

\item If {\tt soln\_in} is specified, then {\tt params\_file} must
be specified.

\item If {\tt soln\_out} is specified, then both {\tt
params\_file} and {\tt soln\_in} must be specified. However, a
dummy filename for {\tt soln\_in} may be used, allowing one to
save an out-file without an in-file.


\end{itemize}

\noindent Regarding usage \#2: This can be used to automatically
generate a valid {\tt params\_file} for SDPLR. The user is
prompted to enter each parameter value, and detailed explanations
of the parameters can be gotten by just typing `{\tt i}'.

\section{Input Formats}

By default, SDPLR accepts the sparse SDPA format, which is
explained at
\begin{center}
% \url{http://www.nmt.edu/~sdplib/FORMAT}
\url{http://euler.nmt.edu/~brian/sdplib/FORMAT}
\end{center}
However, some SDPs are not sparse but still have a great deal of
structure. The SDPLR format, which SDPLR is also capable of
handling, is more or less an extension of sparse SDPA format which
also allows low-rank data matrices to be specified easily.

The structure of the SDPLR format is as follows:

\begin{itemize}

\item The first line contains $m$, the number of constraint
matrices.

\item The second line contains $k$, the number of blocks in the SDP.

\item The next $k$ lines contain the sizes $n_1, \ldots, n_k$ of the
blocks, where a negative $n_j$ indicates a diagonal block.

\item The next line contains $b$, the right-hand side vector, all
on one line.

\item The next line is currently ignored, but should contain a
number. (This line is related to a feature of SDPLR which is in
development. Basically, for +/-1 combinatorial optimization SDPs,
this line will contain the trace of the primal matrix (a
constant), which can be used to provide dual bounds during the
execution of SDPLR.)

\item The remaining portion of the file is divided into $(m+1)k$
sections, giving the $k$ blocks of the $m+1$ objective and constraint
data matrices. The matrices should be listed in order, i.e., the
objective matrix first and then the constraint matrices in order, and
within each matrix, the blocks should be listed in order.

\item Two types of blocks can be specified for the data matrices: sparse
and low-rank.

\item If the $j$-th block of $A_i$ is a sparse data matrix, then the
section is as follows:

\begin{itemize}

\item The first line contains the matrix number $i$ ($i = 0$ for
objective), the block number $j$, the character {\tt s} to indicate
`sparse,' and the number nnz, which is the number nonzeros in the upper
triangular part of $A_i$.

\item The next nnz lines contain the entries in ``$i \ \ j \ \ $
entry'' format, where $i \le j$.

\end{itemize}

\item If the $j$-th block of $A_i$ is a low-rank data matrix of rank
$r$, then a factorization $B D B^T$ of this block must be known such
that $B \in \Re^{n_j \times r}$ and $D \in \Re^{r \times r}$ is
diagonal. Then the section specifying the block is as follows:

\begin{itemize}

\item The first line contains the matrix number (0 for objective), the
block number $j$, the character {\tt l} to indicate `low-rank,' and the
number $r$, which is the rank of the $j$-th block of $A_i$.

\item The next $r$ lines contain the diagonal entries of $D$ in
order.

\item The next $n_j r$ lines contain the entries of $B$ in
column-major format. Note that $B$ is specified as dense.

\end{itemize}

\end{itemize}

A brief example of the SDPLR format is the following, which encodes
the Lov\'asz theta SDP for the 5-cycle (more examples
available on the SDPLR website):
% \newpage
\begin{verbatim}
6
1
5
0.0 0.0 0.0 0.0 0.0 1.0
-1.0
0 1 l 1
-1.0
1.0
1.0
1.0
1.0
1.0
1 1 s 1
1 2 1.0
2 1 s 1
2 3 1.0
3 1 s 1
3 4 1.0
4 1 s 1
4 5 1.0
5 1 s 1
1 5 1.0
6 1 s 5
1 1 1.0
2 2 1.0
3 3 1.0
4 4 1.0
5 5 1.0
\end{verbatim}

%\bibliographystyle{plain}
%\bibliography{../bibliography}

\end{document}
