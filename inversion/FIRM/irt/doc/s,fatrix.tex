% s,fatrix

\subsection{The \fatrix class}

Most iterative algorithms
for image reconstruction
are described conveniently using matrix notation,
but matrices are not necessarily
the most suitable data structure
for actually implementing an iterative algorithm
for large sized problems.
The \fatrix class
provides a convenient bridge
between matrix notation
and practical system models
used for iterative image reconstruction.

Consider the simple iterative algorithm
for reconstructing \x from data \y
expressed mathematically:
\be
\x^{n+1} = \x^n + \alpha \A' (\y - \A\x^n)
,\ee{e,fatrix,iter}
where \A is the \emph{system matrix}
associated with
the image reconstruction problem at hand.
%
If \A is small enough
to be stored as a matrix in \matlab
(sparse or full),
then this algorithm translates
very nicely into \matlab as follows.
%\begin{verbatim}
\be
\ty{x = x + alpha * A' * (y - A * x);}
\ee{e,mat}
%\end{verbatim}
You really cannot get any closer connection
between the math and the program than this!
But these days we often work
with system models
that are too big to store
as matrices in \matlab.
Instead,
the models
are represented by subroutines
that compute the ``forward projection'' operation
$\A \x$
and the ``backprojection operation
$\A' \z$
for input vectors \x and \z respectively.
%
The conventional way
to use one of these systems
in \matlab (or C) would be to rewrite
the above program as follows.
\begin{verbatim}
Ax = forward_project(x, system_arguments)
residual = y - Ax;
correction = back_project(residual, system_arguments)
x = x + alpha * correction
\end{verbatim}
Yuch!
This is displeasing for two reasons.
First,
the code looks a \emph{lot} less like the mathematics.
Second,
you need a different version of the code
for every different system model
(forward/back-projector pair)
that you develop.
Having multiple versions
of a simple algorithm
creates a software maintenance headache.

The elegant solution
is to develop \matlab objects
that know how to perform
the following operations:
\blist
\item
\ty{A * x}
(matrix vector multiplication,
operation \ty{mtimes})
\item
\ty{A'}
(\ty{transpose}), and
\item
\ty{A' * z}
(\ty{mtimes} again,
with a transposed object).
\elist
Once such an object is
defined,
one can use \emph{exactly}
the same iterative algorithm
that one would have used
with an ordinary matrix,
\eg,
\eref{e,mat}.
%
The \fatrix class
provides a convenient mechanism
for implementing
such linear operators.
%
Specifically,
suppose \x is of length 1000
and \y is of length 2000.
Then use the following call:
\[
\ty{
A = Fatrix([2000 1000], system_arguments,
 'forw', @forward_project, 'back', @back_project);
}
\]
The restulting \fatrix object \ty{A}
acts just like a matrix
in most important respects.
In particular,
we can use exactly the same iterative algorithm
\eref{e,fatrix,iter} as before,
because
\ty{A * x}
is handled internally
by calling
\[
\ty{
Ax = forward_project(x, system_arguments)
}
\]
and similarly for
\ty{A' * z}.

Basic operations
like \ty{A(:,7)}
are also implemented,
but nonlinear operations
like \ty{A .^ 1/3}
are not
because those cannot be computed readily
using \ty{forward_project}. 

For examples, see
the \ty{systems} subdirectory of \irt.

On 2007-1-30,
inspired by \ty{bbtools},
I added the following functionality:
\blist
\item
Fatrix object multiplication
(using \ty{Gcascade}): \ty{C = A * B}
\item
Scalar multiplication of Fatrix object
(using \ty{Gcascade}): \ty{B = 7 * A}
\item
Vertical concatenation (using \ty{block_fatrix}):
\ty{A = [A1; A2; A3]}
\item
Horizontal concatenation (using \ty{block_fatrix}):
\ty{A = [A1, A2, A3]}
\elist
One could use \ty{Gcascade} or \ty{block_fatrix}
directly for these operations,
but it looks nicer
and is more ``\matlab like''
to use the new syntax.
